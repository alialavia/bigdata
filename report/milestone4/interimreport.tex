% This file is the Latex source of the Big Data course project
% report. The project contributors are Ali Alavi, Rolf jagerman
% and Ken Tsay.
% The report is written by Ali Alavi, Rolf Jagerman, Ken Tsay.

%
\documentclass{llncs}
%

\usepackage{graphicx} % for importing images
\usepackage{caption}
\usepackage{subcaption} % for subfigures
\captionsetup{compatibility=false} % to make subfigures compatible with template

\usepackage{url} % for URL references
\urlstyle{same}
\usepackage{float} % helps with locating the 
\usepackage[T1]{fontenc}  % providing font encoding
% used for drawing the diagrams
%
\begin{document}
%
\mainmatter              % start of the contributions
\pretolerance=10000  % This avoids long lines
\pagestyle{headings}
%\hyphenation{}

%
\title{Automatic News Generation Based on Twitter}
%
\titlerunning{Automatic News Generation Based on Twitter}  % abbreviated title (for running head)
%                                     also used for the TOC unless
%                                     \toctitle is used
%
\author{Ali Alavi\inst{1} \and Rolf Jagerman\inst{1} \and
Tsay Kai-En\inst{1}}
%
\authorrunning{Ali Alavi, Rolf Jagerman and Tsay Kai-En} % abbreviated author list (for running head)
%
%
\institute{ETH Z\"urich, Z\"urich, Switzerland\\
\email{alavis@ethz.ch, \{rolfj, tsayk\}@student.ethz.ch}
}

\maketitle              % typeset the title of the contribution
%

\section{Abstract}
In this report, we describe our approach towards generating news topics based on twitter data. We tackle this problem by running a stochastic gradient descent classifier on a large set of news articles collected from different news agencies, and then using this classifier to classify twitter posts into three news categories: sports, politics, and technology. Then we tokenize these classified tweets in order to extract names and nouns used in each tweet. Finally we perform a time series analysis on these set of words and recognize top ten trending topics as news-worthy. We present the results in a web based interface. The results, especially in politics and sports category, bear resemblance with the trending topics as reported by news agencies.

\section{Introduction}

The authors were motivated by a simple question: can we generate more accurate and less biased news using twitter data in comparison to traditional news agencies? Would such a system have a potential of becoming an alternative to mainstream news sources? If so, such system can be a more trustable source of unbiased news, which in turn will have immense effect on public awareness, knowledge and discourse.

Although there are many tools and websites, such as Google News, which automatically aggregate and present news articles, their data sources are  mainstream news agencies. We, on the other hand, want to use public posts as our data source, hence using collective knowledge of citizens as our news agency. Hence, the citizens will we be the audience as well as content providers.

In this model, every twitter user can play a small, yet collectively significant role in news gathering, and hence in generating valuable news articles, even without his or her knowledge. This is the main difference between our vision and that of citizen journalism, where citizens intend to produce a news headline, article or story. A concept we would like to call \textit{crowd-reporting}.

There is not much traditional techniques that can be used for realizing such system, since the core concept of this system makes use of big data collected from Twitter, something that no traditional technique can replace. Although we can think of using one or more data sources other than Twitter to achieve similar results.

\section{Performance Measurements and Results}

\section{Conclusion}

\bibliographystyle{plain}
\bibliography{report.bib}

\end{document}