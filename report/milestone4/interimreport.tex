% This file is the Latex source of the Big Data course project
% report. The project contributors are Ali Alavi, Rolf jagerman
% and Ken Tsay.
% The report is written by Ali Alavi, Rolf Jagerman.

%
\documentclass{llncs}
%

\usepackage{graphicx} % for importing images
\usepackage{caption}
\usepackage{subcaption} % for subfigures
\captionsetup{compatibility=false} % to make subfigures compatible with template

\usepackage{url} % for URL references
\urlstyle{same}
\usepackage{float} % helps with locating the 
\usepackage[T1]{fontenc}  % providing font encoding
% used for drawing the diagrams
%
\begin{document}
%
\mainmatter              % start of the contributions
\pretolerance=10000  % This avoids long lines
\pagestyle{headings}
%\hyphenation{}

%
\title{Automatic News Generation Based on Twitter}
%
\titlerunning{Automatic News Generation Based on Twitter}  % abbreviated title (for running head)
%                                     also used for the TOC unless
%                                     \toctitle is used
%
\author{Ali Alavi\inst{1} \and Rolf Jagerman\inst{1} \and
Tsay Kai-En\inst{1}}
%
\authorrunning{Ali Alavi, Rolf Jagerman and Tsay Kai-En} % abbreviated author list (for running head)
%
%
\institute{ETH Z\"urich, Z\"urich, Switzerland\\
\email{alavis@ethz.ch, \{rolfj, tsayk\}@student.ethz.ch}
}

\maketitle              % typeset the title of the contribution
%

\section{Abstract}

\section{Introduction}

\section{Contribution}

\subsection{How does your solution compare to existing work}

Current approaches to classify messages on Twitter mostly focusing on personal data. The complexity of handling massive data could be one possible reason. A few works\cite{Go_Bhayani_Huang_2009}\cite{twitter-classifier} applied different machine learning methods(Naive Bayes Classifier, Support Vector Machine, etc) to classify tweets, but their data size were also relatively smaller comparing with us. Other works such as Trendsmap\cite{Trendmap}, shows you the latest trends from Twitter for anyplace in the world, are doing data analysis on Twitter hashtags instead of the content of tweets. Comparing with these existing works, our data size is larger (1 TB) and we tried not only just classify the twitter data, but also automatically generate key terms of news base on these labeled tweets, and compared them to see if they perform better than the available news agencies. We can not find previous research on this.

\subsection{What is the main new contribution of your approach}



\section{Performance Measurements and Results}

\section{Conclusion}

\bibliographystyle{plain}
\bibliography{report.bib}

\end{document}